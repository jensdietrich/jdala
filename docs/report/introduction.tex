\section{Introduction}
%
Concurrent programming is ubiquitous today, driven by hardware advancements and software demands, but prone to subtle errors that often pass undetected and cause unexpected behaviour~\cite{lu2008learning,musuvathi2008finding,lin2015jacontebe}. Mainstream programming languages—especially those with shared-memory concurrency—leave developers vulnerable to concurrency bugs such as data races, race conditions, and deadlocks, which are notoriously difficult to detect, reproduce, and debug.
%
\dala~\cite{Dala_Paper,DafnyExperience-Noble2024} is a simple, novel model of object-oriented capabilities with dynamic enforcement. \dala allows parts of a program’s state to be protected from mutation (\texttt{imm}), concurrent access across threads (\texttt{local}), or aliased references (\texttt{iso}), thereby helping inhibit data races and related errors. These three capability flavours enforce constraints: immutability prohibits state change, isolation ensures thread exclusivity and non-aliasing, and locality guarantees thread affinity. Violations of these constraints are detected dynamically and halted with informative errors, enabling a fail-fast programming style~\cite{shore2004fail} that improves safety and ability to debug.
%
Work on \dala has been based on extensions to the Grace programming language~\cite{GraceAbsence-Black2012}, a simple, educational language designed with minimal concepts. This prior work has not addressed how the model might be applied to existing, complex, production-oriented languages with real-world concurrency requirements.
%
\jdala is an attempt to augment Java with \dala capabilities: to allow Java objects to be given one of the \dala flavours, and to detect and report violations of their constraints at run time. Unlike static approaches such as Rust’s affine type system or Pony’s actor-based isolation, \jdala uses lightweight annotations and bytecode instrumentation to retroactively enforce capability restrictions within standard Java programs. With only three core annotations, Java developers can sidestep common concurrency pitfalls—such as unsafe aliasing or illegal thread-sharing—while retaining the structure and familiarity of the Java ecosystem.
%
% Jens: shortened
%\jdala permits the programmer to mark some parts of the code with additional protections, while leaving other parts unchanged, enabling incremental adoption with minimal impact on existing code bases. As in many gradual typing systems, \jdala performs dynamic run-time checks to ensure that restrictions are not violated, and halts the program immediately if they are. 
%
\jdala lets the program execute until a visible error occurs, then prevents the effects of that error from propagating further. This fail-fast behaviour assists developers in debugging and prevents concurrency bugs, like race conditions or misuse of immutability, from corrupting the run-time state of the program.

An alternative to implement \dala in Java would have been the use of a pluggable types system like \textit{JavaCOP}~\cite{markstrum2010javacop} or the \textit{checkerframework}~\cite{ernst2010building}. In particular for the \textit{checkerframework}, several checkers similar to \dala (e.g. immutability) already exist. Rules are enforced statically, i.e. at compile time. 
A drawback is the handling of dynamic programming patterns like reflection that are pervasive in Java~\cite{sui2020recall} and notorious difficult to model by static analyses~\cite{ernst2003static,livshits2015defense}, resulting in unsoundness or false alerts (if the analysis over-approximates dynamic language features, e.g. by considering all objects returned by \texttt{Method::invoke} as immutable). \jdala's dynamic approach can avoid this, although at the cost of runtime overhead.

%
The contributions of this paper are: 
(1) A design for \jdala, adding \dala capabilities to Java.
(2) A prototype implementation of \jdala using annotations and bytecode instrumentation.
(3) Examples of how \jdala can avoid common bugs.
