\documentclass[JDala.tex]{subfiles}
\graphicspath{{\subfix{../images/}}}
\begin{document}

\todo[inline]{@James @Michael  - less than one page, talk about PL aspect, and perhaps fail fast}

Concurrent programming is ubiquitous today, driven by hardware advancements
and pressures, but prone to subtle errors that pass undetected and cause
unexpected behaviour.
%
\dala~\cite{Dala_Paper,DafnyExperience-Noble2024} is a simple, novel model of object-oriented capabilities with dynamic
enforcement that allows parts of a program's state to be protected from
mutation (\texttt{imm}), concurrent access across threads (\texttt{local}),
or aliased references (\texttt{iso})
to inhibit data races.
%
Work on \dala has been based on extensions of the Grace programming
language~\cite{GraceAbsence-Black2012}, a simple educational language with
minimal concepts, but has not touched on how the model can be applied to
existing, complex, production languages.
%
\jdala is an attempt to augment Java with \dala capabilities:
%
to allow Java objects to be given one of the \dala capabilities,
and to detect and report violations of these restricted capabilities when
they occur at run time.

\jdala permits the programmer to mark some parts of the code with
additional protections, while leaving other parts of the code unchanged
and having minimal impact on an existing code base.
%
As in many gradual typing approaches, \jdala performs run-time checks to
ensure that restrictions are not violated, and prevents the program from
continuing if they are.
%
This approach lets the program run until a visible error occurs, then
prevents the effects of that error from propagating further, assisting
debugging while preventing the impacts of, for example, data races from
affecting the run-time results of the program.

The contributions of this paper are:
\begin{itemize}
  \item A design for \jdala, adding \dala capabilities to Java
  \item A prototype implementation of \jdala that uses annotations and bytecode instrumentation
  \item Examples of how \jdala can avoid concurrency bugs like deadlocks
\end{itemize}


\end{document}