\documentclass[conference]{IEEEtran}
\IEEEoverridecommandlockouts
% The preceding line is only needed to identify funding in the first footnote. If that is unneeded, please comment it out.
%Template version as of 6/27/2024

\usepackage{cite}
\usepackage{amsmath,amssymb,amsfonts}
\usepackage{algorithmic}
\usepackage{graphicx}
\usepackage{textcomp}
\usepackage{xcolor}
\usepackage{subfiles}
\usepackage{todonotes}
\usepackage{listings}
\usepackage{xspace}

\newcommand{\dala}{Dala\xspace}
\newcommand{\jdala}{JDala\xspace}
\newcommand{\Immutable}{\texttt{@Immutable}\xspace}
\newcommand{\Local}{\texttt{@Local}\xspace}
\newcommand{\Isolated}{\texttt{@Isolated}\xspace}

\def\BibTeX{{\rm B\kern-.05em{\sc i\kern-.025em b}\kern-.08em
    T\kern-.1667em\lower.7ex\hbox{E}\kern-.125emX}}


\lstdefinestyle{mystyle}{
	basicstyle=\ttfamily\scriptsize,
	breakatwhitespace=false,         
	breaklines=true,                 
	captionpos=b,                    
	keepspaces=true,                 
	numbers=left,                    
	numbersep=5pt,                  
	showspaces=false,                
	showstringspaces=false,
	showtabs=false,                  
	tabsize=2,
	frame=single
}

\lstset{style=mystyle}


\begin{document}
	
	

\title{JDala - A Simple Capability System for Java*\\
\thanks{TODO: @James.}
}

\author{\IEEEauthorblockN{1\textsuperscript{st} Quinten Smit}
\IEEEauthorblockA{\textit{School of Engineering and Computer Science} \\
\textit{Victoria University Of Wellington}\\
Wellington, New Zealand \\
smitquin@myvuw.ac.nz}
\and
\IEEEauthorblockN{2\textsuperscript{nd} Jens Dietrich}
\IEEEauthorblockA{\textit{School of Engineering and Computer Science} \\
\textit{Victoria University Of Wellington}\\
Wellington, New Zealand \\
jens.dietrich@vuw.ac.nz}
\and
\IEEEauthorblockN{3\textsuperscript{rd} Michael Homer}
\IEEEauthorblockA{\textit{School of Engineering and Computer Science} \\
\textit{Victoria University Of Wellington}\\
Wellington, New Zealand \\
michael.homer@vuw.ac.nz}
\and
\IEEEauthorblockN{4\textsuperscript{th} Given Name Surname}
\IEEEauthorblockA{\textit{dept. name of organization (of Aff.)} \\
\textit{name of organization (of Aff.)}\\
City, Country \\
email address or ORCID}
\and
\IEEEauthorblockN{5\textsuperscript{th} James Noble}
\IEEEauthorblockA{\textit{dept. name of organization (of Aff.)} \\
\textit{name of organization (of Aff.)}\\
City, Country \\
email address or ORCID}
}

\maketitle

\begin{abstract}


\dala is a novel capability-based programming model that ensures data-race freedom while also supporting efficient inter-thread communication. While \dala has been designed top inform the design of future programming languages, the question arises whether existing languages can be retrofitted with \dala capabilities. We report such a design called \jdala. In \jdala,  \dala capabilities are added to Java using annotations and interpreted using bytecode instrumentation. With some examples we demonstrate that by adding three simple annotations to the language, we can avoid concurrency bugs like deadlocks and unexpected program behaviour resulting from shallow immutability of Java standard library APIs. 

\end{abstract}

\begin{IEEEkeywords}

\end{IEEEkeywords}

\section{Introduction}

\subfile{introduction}

\section{Related Work}

\subfile{related_work}

\section{Design}

\subfile{design}

\section{examples / evaluation?}

\subfile{examples}

\section{Conclusion}

\subfile{conclusion}

\bibliographystyle{plain}
\bibliography{references}

\end{document}
