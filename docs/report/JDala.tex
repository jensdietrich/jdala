\documentclass[conference]{IEEEtran}
\IEEEoverridecommandlockouts
% The preceding line is only needed to identify funding in the first footnote. If that is unneeded, please comment it out.
%Template version as of 6/27/2024

\usepackage{cite}
\usepackage{amsmath,amssymb,amsfonts}
\usepackage{algorithmic}
\usepackage{graphicx}
\usepackage{textcomp}
\usepackage{xcolor}
\usepackage{subfiles}
\usepackage{todonotes}
\usepackage{listings}
\usepackage{xspace}
\usepackage{url}

\newcommand{\dala}{Dala\xspace}
\newcommand{\jdala}{JDala\xspace}
\newcommand{\Immutable}{\texttt{@Immutable}\xspace}
\newcommand{\Local}{\texttt{@Local}\xspace}
\newcommand{\Isolated}{\texttt{@Isolated}\xspace}
\newcommand{\portal}{\texttt{Portal}\xspace}

\def\BibTeX{{\rm B\kern-.05em{\sc i\kern-.025em b}\kern-.08em
    T\kern-.1667em\lower.7ex\hbox{E}\kern-.125emX}}


\lstdefinestyle{mystyle}{
	basicstyle=\ttfamily\scriptsize,
	breakatwhitespace=false,         
	breaklines=true,                 
	captionpos=b,                    
	keepspaces=true,                 
	numbers=left,                    
	numbersep=5pt,                  
	showspaces=false,                
	showstringspaces=false,
	showtabs=false,                  
	tabsize=2,
	frame=single
}

\lstset{style=mystyle}


\begin{document}

\title{JDala - A Simple Capability System for Java\thanks{This work is supported in part by the Royal Society of New
  Zealand Te Ap\={a}rangi Marsden Fund Te P\={u}tea Rangahau a Marsden
  grant CRP2101.}
}


\author{
	\IEEEauthorblockN{Quinten Smit\IEEEauthorrefmark{1}, Jens Dietrich\IEEEauthorrefmark{1}, Michael Homer\IEEEauthorrefmark{1}, Andrew Fawcett\IEEEauthorrefmark{1}, James Noble\IEEEauthorrefmark{2}}
	\IEEEauthorblockA{\IEEEauthorrefmark{1}Victoria University of Wellington, School of Engineering and Computer Science, Wellington, New Zealand
		\\smitquin@myvuw.ac.nz,\{jens.dietrich,michael.homer,andrew.fawcet\}@vuw.ac.nz}
	\IEEEauthorblockA{\IEEEauthorrefmark{2}Creative Research and
          Programming, Wellington, New Zealand, kjx@acm.org}
}
\maketitle

\begin{abstract}
\dala is a novel capability-based programming model that ensures data-race freedom while also supporting efficient inter-thread communication. While \dala has been designed to inform the design of future programming languages, the question arises whether existing languages can be retrofitted with \dala capabilities. We report such a design called \jdala. In \jdala,  \dala capabilities are added to Java using annotations and interpreted using bytecode instrumentation. With some examples we demonstrate that by adding three simple annotations to the language, we can avoid concurrency bugs like deadlocks and unexpected program behaviour resulting from shallow immutability of Java standard library APIs. \\
\jdala demo scenarios \url{https://youtu.be/QddK1q35h-U}
\end{abstract}

\begin{IEEEkeywords}
object capabilities, java, instrumentation, concurrency, immutability 
\end{IEEEkeywords}

\section{Introduction}
%
Concurrent programming is ubiquitous today, driven by hardware advancements and software demands, but prone to subtle errors that often pass undetected and cause unexpected behaviour~\cite{lu2008learning,musuvathi2008finding,lin2015jacontebe}. Mainstream programming languages—especially those with shared-memory concurrency—leave developers vulnerable to concurrency bugs such as data races, race conditions, and deadlocks, which are notoriously difficult to detect, reproduce, and debug.
%
\dala~\cite{Dala_Paper,DafnyExperience-Noble2024} is a simple, novel model of object-oriented capabilities with dynamic enforcement. It allows parts of a program’s state to be protected from mutation (\texttt{imm}), concurrent access across threads (\texttt{local}), or aliased references (\texttt{iso}), thereby helping inhibit data races and related errors. These three capability flavours enforce constraints: immutability prohibits state change, isolation ensures thread exclusivity and non-aliasing, and locality guarantees thread affinity. Violations of these constraints are detected dynamically and halted with informative errors, enabling a fail-fast programming style~\cite{shore2004fail} that improves safety and ability to debug.
%
Work on \dala has been based on extensions to the Grace programming language~\cite{GraceAbsence-Black2012}, a simple, educational language designed with minimal concepts. However, this prior work has not addressed how the model might be applied to existing, complex, production-oriented languages with real-world concurrency requirements.
%
\jdala is an attempt to augment Java with \dala capabilities: to allow Java objects to be given one of the \dala flavours, and to detect and report violations of their constraints at run time. Unlike static approaches such as Rust’s affine type system or Pony’s actor-based isolation, \jdala uses lightweight annotations and bytecode instrumentation to retroactively enforce capability restrictions within standard Java programs. With only three core annotations, Java developers can sidestep common concurrency pitfalls—such as unsafe aliasing or illegal thread-sharing—while retaining the structure and familiarity of the Java ecosystem.
%
\jdala permits the programmer to mark some parts of the code with additional protections, while leaving other parts unchanged, enabling incremental adoption with minimal impact on existing code bases. As in many gradual typing systems, \jdala performs dynamic run-time checks to ensure that restrictions are not violated, and halts the program immediately if they are. This approach lets the program execute until a visible error occurs, then prevents the effects of that error from propagating further. This fail-fast behaviour assists developers in debugging and prevents concurrency bugs, like race conditions or misuse of immutability, from corrupting the run-time state of the program.
%
The contributions of this paper are:
\begin{itemize}
\item A design for \jdala, adding \dala capabilities to Java
\item A prototype implementation of \jdala using annotations and bytecode instrumentation
\item Examples of how \jdala can avoid concurrency bugs such as data races and deadlocks
\end{itemize}


\section{Related Work}
	
\todo[inline]{@James @Michael}

\todo[inline,color=pink]{@James has added generic background stuff on
  concurrency cirectly below - needs to be trimmed hard.}


Concurrent programming is hard \cite{lea98,concurrentHard}
%
%Concurrent programming language design is even harder \cite{HoareHints}.
%
%JavaConcur
%Concurrent programming language design is even harder \cite{HoareHints}.
%
because concurrent programs do two or more things at the same
time. Even if one thread of execution is perfectly correct, another
can interfere with its behaviour, corrupting data, triggering
deadlock, or crashing the whole system \cite{fonesca2010}.
%
%
The most important kind of interference between threads is
\textit{data races} --- when one thread writes to shared memory that
another thread is reading or writing.
%
In a data race, values stored into or retrieved from memory can be
essentially arbitrary.
% The problem with data races is
%that values stored into memory, and the values returned by any reads,
%may be essentially arbitrary.
%, depending on the semantics of the
%particular language, its implementation, and even the underlying CPU
%and memory architecture!
%
%
%
%
%
This is exacerbated because concurrent programs are typically written in low-level languages
such as C, C$++$, BLISS-32, or Swift, which provide
few correctness guarantees, making it easy to write \textbf{wrong}
concurrent programs
\cite{CCppSCAM2014,LuSurvey2008}.
%
%\cite{stefikConcur2020,aba,raffi2020}
%
%% still like this, don't want to lose it!!!
%%
%% Suggestions that functional programming can solve all problems of
%% concurrent programming in the object-oriented mainstream have been
%% rebutted by pointing to a simple key problem: \textit{``assignment is
%%   faster than copying''} to paraphrase Doug Lea
%% \cite{AssignmentVSCopying}. In a world of mathematical abstractions,
%% we can (probably) ignore the physics and thermodynamics underlying our
%% programs: we can copy any amount of information any number of times on
%% any number of processors all for free, so we never need to mutate any
%% existing information, and we never have to face the problems of
%% synchronisation and inference between threads.
%% %
%% For the rest of us, our programs end up embodied in physical models
%% running on physical processors \cite{betabook}, where copying a
%% gigabyte of information unavoidably takes a million times as much work, uses
%% a million times more power, allocates a million times as much space,
%% and generates a million times more heat, than overwriting one
%% kilobyte. We cannot ignore the differences that leak out between our
%% logical, immutable maps, and their underlying implementation
%% territories of memory locations that store data and the threads that
%% animate them \cite{CopyExpensive}.
%
% SOB...
%
%Welcome to the desert of the real
%\cite{baurdillard,theMatrix,zizek}.
%
%

A range of programming techniques have been developed to avoid these problems
\cite{lea98,concurrentHard}, aiming to provide \textit{safe
  concurrency} where data races are prevented either by the design  of the
programming language itself or its associated tools.
%  Unfortunately, while 
%concurrent programming is hard, safe concurrent programming language
%design turns out to be even harder \cite{RustBook,EncoreTS}. 
%
For example, full-scale proof systems
%\cite{vcc09}
such as concurrent separation logics\cite{JacobsEA05,chalice},
Rely/Guarantee\cite{jonesTOPLAS83,MPC-Staden15,concur2007} and
Deny/Guarantee\cite{DenyGuarantee} reasoning, and IRIS\cite{dd}
can guarantee correctness for programs in almost any language,
but are too specialised for developers to use in practice
\cite{fonesca2017,shriramFormal2019}.

More pragmatically,
contemporary programming languages such as
Rust\cite{RustBook}, Pony\cite{PonyTS},
Encore\cite{EncoreTS},
%Singularity\cite{Singularity},
%Deterministic Parallel Java\cite{DPJ},
%Safe C\verb+#+\cite{GordonPPBD12},
Obsidian~\cite{aldrichObsidianStudy2020}, and
Verona~\cite{Verona} 
have demonstrated the efficacy of \textit{static
  ownership}\cite{ClaPotNobOOPSLA98,NobPotVitECOOP98} %%% BoyLeeRinOOPSLA02,ClaPotNobOOPSLA98 
%and capability systems
to ensure
concurrent programs are safe.
%: Rust in particular has been adopted by
%Microsoft\cite{RustPopular,MSRust}.
By keeping track of each object's
ownership, these languages determine when an object may be used
or changed, and what other code can be affected by changes.
%
%
%% Concurrent programming languages face three problems:
%% %
%% \textbf{Correctness}: concurrent threads must only interact with each other
%% in well understood ways \cite{MPC-Staden15,concur2007,chalice}.
%% %
%% % interference between threads must be prevented.
%% %
%% \textbf{Simplicity}: programmers must be able to read, write, and understand
%% safe, efficient programs \cite{lea98,stefikConcur2020,godefroid2008}.
%% %
%% \textbf{Performance}: 
%% concurrent programs should run efficiently
%% (or else why not just write a non-current program\cite{TeslaRewrite,RustRewrite}).
%% %
%
%
While much simpler than full-scale program proof systems, these static
concurrent 
languages rely on complicated static rules and
restrictions, with many different capability annotations and ownership
parameterisations that programmers find hard to learn and use
correctly \cite{LearnRust,VizRust,HardRust}: they support writing
correct and efficient programs, but they are still hard to understand
\cite{SafeRust,FightRust} for a number of
reasons. First, their design must be conservative, banning not just
all concurrent programs that are \textit{actually} unsafe, but a large
number of data-race free programs as well, resulting in
many false positive errors for
problems that will never arise in practice.
%For
%example, Rust's version of ownership types\cite{RustBook} bans even
%such common idioms such as circular or doubly-linked lists. 
Second,
programmers typically have to annotate their programs to give the
ownership and capability checkers the information they need,
with detailed bespoke annotations setting how each object or
variable is used.
% --- so
%rather than declaring an input stream ``\verb+in+'', programmers need
%to write complex expressions such as ``\verb+in : &mut InStream<'a>+''
%(where ``\verb+&mut+'' indicates that ``\verb+in+'' is a mutable
%reference, ``\verb+InStream+'' indicates that ``\verb+in+'' refers to
%an input stream, and  ``\verb+<'a>+'' is a lifetime (a.k.a. ownership)
%parameter indicating the originating scope of the input stream.
Finally, these annotations are required \textit{throughout}
the program,  even if only a small part is actually concurrent, or is
otherwise likely to cause critical errors.
% --- in Rust, an inflight
%entertainment system would have to be engineered to the same level of
%quality as a critical flight control system, even though the risks and
%requirements for each system are very different.


%how to bring in actors etc?
%not sure about this bit
%
%
%
%% Application programmers are increasingly moving towards
%% \textit{dynamic} languages such as Python, R, and Julia
%% \cite{dynamic}.  These languages make it easier to write sequential
%% code, but generally provide no more support for wiring concurrent
%% code than sequential systems programming languages (and are are much
%% less efficient).  The Julia documentation, for example, says that
%% \textit{`` You are entirely responsible for ensuring that your program
%%   is data-race free, and nothing promised here can be assumed if you
%%   do not observe that requirement. The observed results may be highly
%%   unintuitive ''}
%% \cite{https://docs.julialang.org/en/v1/manual/multi-threading/}
%% \textsf{Q - do I want to say ``data race freedom?''.  or what?.}
%

On the other hand,
%meanwhile
\textit{dynamically} concurrent languages like
JavaScript\cite{JSVAts},
%Erlang\cite{Erlang},
E\cite{MillerPhD}, and
AmbientTalk\cite{AmbientTalk} support simple,
concurrency-safe programming by design, ruling out whole classes of
bugs.
Programmers don't need to annotate their code,
%(an input stream could be just declared as ``\verb+in+'')
making programs easier to read and write.
Unfortunately, these kinds of dynamic approaches achieve safety and simplicity
by sacrificing performance, in particular by banning efficient
communication between concurrent threads and requiring
expensive alternatives~\cite{destructive-read,CASTEGREN2018130,JSproxies,AmbientTalk}.
%It is not possible to transfer ownership of an object:
%instead, the objects must either be copied
%\cite{destructive-read}
%%
%%(which while simplifying
%%garbage collection may be expensive \cite{CopyExpensive} and loses
%%object identity),
%(taking significant time and memory, and losing object identity)
%or proxied back to their originating thread
%\cite{CASTEGREN2018130,JSproxies,AmbientTalk}
%%(which makes it hard to reason about performance unless it is clear
%%that an operation is asynchronous).
%(greatly delaying every access from all other threads).
Some recent dynamically concurrent research languages do e.g.\ permit
information to be transferred or shared directly between threads,
but either carry out runtime checks to ensure threads do not interfere
\cite{Daloze2016,Daloze2018}
or omit those checks, permitting data races, and giving up on safety
\cite{GoConcurBugs2019,raffi2020}.



\todo[inline,color=pink]{
  @James hack at more specific background:
  Do you want to move some of this later ---
  the bit about pyrona/ lungfish talks about Dala
}

In contemporary industrial software development, a range of techniques
for dynamic race detection are used in practice across a range of
programming languages
\cite{hong2015survey,o2003hybrid,cai2021sound,TSanRV2011,DBLP:conf/rv/EricksonFM12}.  Go, for
example, which includes features
encouraging the use of many lightweight threads, but without any control
or partitioning of threads' memory accesses,  encourages the use
of a race detector as a normal part of the development toolchain, and
even for \textit{``limited production use''}
\cite{GoCACM2022,TSanRV2011}.

The issue with race detectors is that they detect (potential)
data races, but don't explain why a particular data race
occurs, and thus what developers need to do to fix it.  This is
because data races depend on the relationship between a program's threads and
its data, and without some kind of ownership system to make that
relationship explicit, there is little more they can do.

Dynamic checkers and visualisers have been developed for ownership
systems, as research tools \cite{hill:2002:jvlc,MitchellECOOP2009} and
dynamic ownership languages \cite{dynamicOwn,dynamicAlias}. The last few
years have seen an 
increasing interest in visualisations for Rust \cite{RustBook}, as the only
language with ownership (albeit static ownership in Rust's case) to
achieve some level of widespread adoption.

Alongside Rust, a range of other languages have atempted to
incorporate ownership to deal with concurrency, including Pony
\cite{Pony}, Verona \cite{VeronaMemory2023,VeronaConcur2023}, and a
recent proposal for Swift \cite{gallifrey-pldi2022}. In many ways the
closest to Dala is a concurrent proposal for Python, also based on
dynamic ownership checking to maintain an explicit relationship
between objects and the threads that can access them
\cite{pyrona2025}. This proposal, \textit{Lungfish}, organises objects
in regions based on their ownership. Lungfish supports equivalents of
Dala's immutable, isolated, and shared objects, along with a fourth
category, ``cowns'' (pronounced ``cones'') for objects guarded by a
lock.




\documentclass[../JDala.tex]{subfiles}
\graphicspath{{\subfix{../images/}}}
\begin{document}


\end{document}


\section{The Design of \jdala}
\label{sec:design}	


\subsection{Capabilities as Annotations}
\label{subsection:capabilitesAsAnnotations}

In \jdala, the \dala capabilities are represented using three dedicated annotation types \Immutable, \Isolated and \Local.  Global data structures (maps that are static members of the \texttt{JDala} class) are used to track objects with the respective capabilities.   

Since in Java  objects cannot be annotated directly, the association between objects and object capabilities is achieved by annotating local variables pointing to objects.  This is illustrated in Listing~\ref{listing:annotation}.  In line 1, the newly created \texttt{List} instance is marked as immutable.  A second list created in line 2 is not marked as immutable at the allocation site, but later. This is achieved by an assignment to the annotated variable \texttt{list3}. 


\begin{lstlisting}[language=Java, caption=Associating objects with capabilities, label=listing:annotation]
@Immutable List list = new ArrayList();
List list2 = new ArrayList();
@Immutable List list3 = list2;
\end{lstlisting}


\subsection{Enforcing the Semantics of Capabilities}

The semantics of the respective capabilities are implemented using instrumentation that injects code to enforce them. 
\jdala uses ASM~\cite{bruneton2002asm} for this purpose, and the project builds an agent that can then be attached to any Java application either statically (using the \texttt{-javaagent} argument) or dynamically.
The instrumentation controls the heap of an instrumented application by maintaining safe objects~\footnote{\textit{Safe} objects are objects annotated using either \Immutable, \Isolated or \Local, all other objects are referred to as \textit{unsafe} objects.} in global data structures, and intercepting and checking bytecode instructions reading and writing fields (i.e. traversing and manipulating the heap).  Instructions for reference type fields, special instructions for fields with one of the various Java primitive types, and instructions to access array slots are all instrumented. 

The injected code broadly falls into two categories: \textit{registration} and \textit{enforcement}. When annotated references to objects are encountered, the corresponding objects are registered as immutable, local or isolated in static maps maintained by the \texttt{JDala} class. In case of immutability, referenced objects are registered as immutable as well. For this purpose, a simple reflection-based heap traversal is performed on the object to be registered as immutable.  If already registered objects are re-registered with a weaker capability~\footnote{The capabilities considered here form a hierarchy, see \cite{Dala_Paper} for details.}, a \texttt{DalaRestrictionException} is raised.  This functionality is implemented in various \texttt{JDala:register*} methods.

When fields of registered objects are accessed, the injected code invokes check methods \texttt{JDala:validate*} to enforce the capability contract. Violations are signalled by raising a \texttt{DalaCapabilityViolationException} exception.


\subsection{Object vs Class-Based Capabilities}


\jdala provides two mechanisms to define the capabilities. The primary method uses the object-level annotations discussed in Section~\ref{subsection:capabilitesAsAnnotations}. This method assigns a capability to an object when a variable pointing to it is annotated. Any future objects that are stored in the local variable after the object that has been annotated must once again have an annotation present to be assigned a capability or they will be considered unsafe.
Once an object has been assigned a capability, it retains that capability—or a stricter one—for its entire lifetime.

A secondary mechanism applies only to \Immutable capabilities and involves defining immutable classes globally. Classes listed in \texttt{resources/immutable-classes.txt} are automatically treated as immutable. This approach is suitable for Java classes that are intrinsically immutable, such as \texttt{String}, \texttt{Boolean}, \texttt{Integer}, and \texttt{Byte}. Note that all primitive types are treated as \texttt{@Immutable} by default.

Recognizing these classes as \Immutable is essential, as it enables them to be safely included within \Local or \Isolated objects without requiring explicit annotation. Without this class-wide designation, such intrinsically immutable types—despite their known immutability —would otherwise be treated as unsafe, potentially restricting their use in contexts where immutability is a requirement.

% thanks @Quinten, this work now ! 


\subsection{Object Initialisation Protocol}

Constructors provide two unique challenges. The first is caused by a Java bytecode optimisation that allows object fields to be set before an object's constructor is called. This isn't allowed in Java source code \footnote{Except JEP447 (\url{https://openjdk.org/jeps/447}) which allows statements before \texttt{super(...)}, JEP447 is current at preview stage.} However, compilers  can still generate such bytecode. At this early stage in the constructor, the object has not yet completed initialisation and does not fully extend  \texttt{Object}. In the Java bytecode, this incomplete state is represented using the special \texttt{UninitializedThis} value instead of the standard \texttt{this} reference. As a result, it cannot yet be treated as a fully valid object, which complicates instrumentation and capability tracking during construction.
To deal with this special case, \jdala 's instrumentation checks whether field access occurs in a constructor, and if so will temporary store values to be checked in local variables. As soon as the super-constructor is called, \jdala will perform all of the validation checks at once for those values. This means that in some cases the line numbers in stack traces in \jdala exceptions created when those checks fail might be off by a few lines. 

% J > Q : I am not sure whether this adds a lot of value, but feel free to bring it back if you feel it does
%Importantly the amount of actions that can be performed in a constructor and before a super constructor is very limited, so errors will still occur before the object can properly used.

The second challenge is related to immutable classes. Unlike annotated objects, which are registered as immutable only after their construction, globally defined immutable classes are considered immutable by default and are therefore subject to capability enforcement from the outset. This creates a complication: during construction, these classes must be allowed to modify their internal state, but doing so must not compromise the overall soundness of the system, particularly with respect to other objects accessed within the constructor.
To address this, \jdala introduces a targeted exception to its enforcement rules: it permits an object to modify its own fields within its own constructor, regardless of whether it is marked as \texttt{@Immutable}. This exemption allows immutable objects to be initialized correctly while preserving capability safety for interactions with other objects during construction.
%\todo[inline]{J $>$ Q: can you please add a short paragraph here ? We do have space for a short code listing. }
%\todo[inline, color=cyan]{Q $>$ J: Sorry didn't quite keep it short, it might be able to be trimed a little bit by doing less detail}
% J>Q: great, I did some cosmetic changes



\subsection{Static Fields}
Static fields are shared across all instances of a class and are therefore considered class-level, rather than object-level. Modifications to static fields are not captured by \jdala's object instrumentation. 
%As such, developers must take extra care when working with static data in the context of \dala capabilities.

\subsection{Memory Leak Prevention}

A key implication of using global data structures for capability tracking is the potential for memory leaks. Because references are stored in maps accessed through static variables, those objects may become eligible for garbage collection. To mitigate this, weak references are employed within these collections, allowing unused objects to be reclaimed by the JVM's garbage collector when no strong references remain.


\subsection{Reflection Support}

Java reflection can be used to bypass conventional field access patterns, posing a challenge to capability enforcement. To address this, \jdala instruments the \texttt{Field::get} and \texttt{Field::set} methods. This ensures that field modifications performed via reflection still trigger the appropriate capability checks.

There are other dynamic programming patterns that could be used to bypass the capability contracts. In particular, \jdala does not currently restrict the use of \texttt{Unsafe}~\cite{mastrangelo2015use}. However, direct use of \texttt{Unsafe} by applications is discouraged and restricted in newer versions of Java.


\subsection{Modelling the Transfer of \Isolated objects}

\dala allows for isolated objects to be transferred between threads.
To move isolated objects between threads \dala outlines the use of a consume method that allows only one reference to an object. 
Any isolated object that moves from one thread to another will lose its affiliation with the first thread. This principle can be recreated in Java by setting a reference to \texttt{null} after an object has moved. This could lead to a non-descriptive \texttt{NullPointerException} later in the code.

For \Isolated objects \jdala allows multiple references to exist in one thread, with checks being performed that the object remains in its associated thread. \Isolated objects are initially associated with the thread they were created in. For transferring an \Isolated object to a new thread a dedicated \textit{portal object} must be used. Portal object protocols have to be defined by specifying methods for objects to enter a portal (therefore being disconnected from the current thread) and methods to exit a portal (therefore becoming associated with the current thread).  While it is possible to define portal objects by annotating objects, with properties to define the respective enter and exit methods, we deemed this too complex. Most transfers follow patterns where portal objects are instances of dedicated \textit{portal classes} like blocking queues.  
These classes are defined in the resource \texttt{portal-classes.json}, a classical example is\texttt{ java.util.concurrent.BlockingQueue} with \texttt{put/\-offer/\-add} entry  and \texttt{poll/\-take/\-remove} exit methods. Defining a particular type as a portal also applies to its subtypes. 

If an isolated object enters a portal it goes into a \textit{transfer state}. In this state the object cannot be accessed \footnote{Object fields can neither be read nor mutated.} by any thread. The object can then leave the portal via one of the predefined exit methods and at this point, it leaves the transfer state and becomes owned by the thread invoking the exit method.


\subsection{Instrumentation Scope and Shading}

Common challenges when building agents are self-instrumentation and instrumentation of internal utility classes, such as the data structures used to track capability-annotated objects. Instrumenting these classes directly could result in unintended recursive instrumentation cycles, potentially compromising runtime stability. To address this, \jdala uses \textit{shading} to incorporate private, namespaced copies of \texttt{org.json}, \texttt{org.plumelib.util}, and a concurrency map wrapper from \texttt{java.util.Collections}. These shaded versions are explicitly excluded from instrumentation. To maintain soundness, these shaded classes is restricted exclusively to the \texttt{JDala} class and should not be accessed by other parts of the application. 

In addition, several \texttt{--add-opens} flags have been introduced to the project's Maven configuration. These flags explicitly grant reflective access to internal Java platform modules that are otherwise inaccessible under the Java Module System. This access is essential for \jdala to perform bytecode instrumentation, intercept field and method accesses, and monitor runtime behaviour across a wide range of classes. Without these flags, the agent would encounter \texttt{IllegalAccessException} at runtime, or fail to apply the necessary instrumentation to enforce capability semantics reliably.

%\subsection{Java Bytecode Instrumentation}
%\jdala employs the ASM framework to instrument Java bytecode at runtime. This allows interception of field and array access operations at the bytecode level. Specifically, instructions like \texttt{getField}, \texttt{putField}, and array access opcodes are intercepted and augmented with capability checks.


% \section{examples / evaluation?}
\section{Usage and Evaluation}
\label{sec:useandevaluate}	


\subsection{Building and Using \jdala}

\jdala can be obtained from GitHub by cloning \url{https://github.com/jensdietrich/jdala/}.


\jdala requires two rounds of compiling, the first one to build the agent, and the second to build the code that is used. The first build creates the agent but skips the tests:

\texttt{mvn clean package -DskipTests}

The reason is that most tests need the agent to function. Tests can be executed in a second build that uses the Java agent created by the first build as value of the \texttt{-javaagent} JVM argument.

\subsection{Evaluation}


The tool is designed to be a Technology Readiness Level 4~\cite{mankins1995technology} and has not been fully evaluated against real world data. Evaluation against relevant  benchmarks, such as \textit{Jacontebe}~\cite{lin2015jacontebe} is future work. This would require the manual annotation of benchmark code. There is however a comprehensive test suite included in the project.  



\section{Conclusion}

We have presented \jdala, a proof-of-concept implementation of the \dala capability model, built atop the Java programming language.
The implementation demonstrates that novel programming language concepts can be retrofitted into existing mainstream languages, thereby promoting their broader adoption. The \jdala framework provides a platform for experimenting with capability-based security in the context of real-world applications.

 


\bibliography{references,Case}
\bibliographystyle{plain}

\end{document}
