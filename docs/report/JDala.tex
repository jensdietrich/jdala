\documentclass[conference]{IEEEtran}
\IEEEoverridecommandlockouts
% The preceding line is only needed to identify funding in the first footnote. If that is unneeded, please comment it out.
%Template version as of 6/27/2024

\usepackage{cite}
\usepackage{amsmath,amssymb,amsfonts}
\usepackage{algorithmic}
\usepackage{graphicx}
\usepackage{textcomp}
\usepackage{xcolor}
\usepackage{subfiles}
\usepackage{todonotes}
\usepackage{listings}
\usepackage{xspace}

\newcommand{\dala}{Dala\xspace}
\newcommand{\jdala}{JDala\xspace}
\newcommand{\Immutable}{\texttt{@Immutable}\xspace}
\newcommand{\Local}{\texttt{@Local}\xspace}
\newcommand{\Isolated}{\texttt{@Isolated}\xspace}
\newcommand{\portal}{\texttt{Portal}\xspace}

\def\BibTeX{{\rm B\kern-.05em{\sc i\kern-.025em b}\kern-.08em
    T\kern-.1667em\lower.7ex\hbox{E}\kern-.125emX}}


\lstdefinestyle{mystyle}{
	basicstyle=\ttfamily\scriptsize,
	breakatwhitespace=false,         
	breaklines=true,                 
	captionpos=b,                    
	keepspaces=true,                 
	numbers=left,                    
	numbersep=5pt,                  
	showspaces=false,                
	showstringspaces=false,
	showtabs=false,                  
	tabsize=2,
	frame=single
}

\lstset{style=mystyle}


\begin{document}
	
	

\title{JDala - A Simple Capability System for Java*\\
\thanks{TODO: @James.}
}

\author{
	\IEEEauthorblockN{Quinten Smit\IEEEauthorrefmark{1}, Jens Dietrich\IEEEauthorrefmark{1}, Michael Homer\IEEEauthorrefmark{1}, Andrew Fawcett\IEEEauthorrefmark{1}, James Noble\IEEEauthorrefmark{2}}
	\IEEEauthorblockA{\IEEEauthorrefmark{1}Victoria University of Wellington, School of Engineering and Computer Science, Wellington, New Zealand
		\\smitquin@myvuw.ac.nz,\{jens.dietrich,michael.homer,andrew.fawcet\}@vuw.ac.nz}
	\IEEEauthorblockA{\IEEEauthorrefmark{2}Creative Research and Programming, Wellington, New Zealand
		\\kjx@acm.org}
}


\maketitle

\begin{abstract}


\dala is a novel capability-based programming model that ensures data-race freedom while also supporting efficient inter-thread communication. While \dala has been designed to inform the design of future programming languages, the question arises whether existing languages can be retrofitted with \dala capabilities. We report such a design called \jdala. In \jdala,  \dala capabilities are added to Java using annotations and interpreted using bytecode instrumentation. With some examples we demonstrate that by adding three simple annotations to the language, we can avoid concurrency bugs like deadlocks and unexpected program behaviour resulting from shallow immutability of Java standard library APIs. 

\end{abstract}

\begin{IEEEkeywords}
object capabilities, java, instrumentation, concurrency, immutability 
\end{IEEEkeywords}

\section{Introduction}

\subfile{introduction}

\section{Related Work}

\subfile{related_work}

\section{Design}

\subfile{design}

\section{examples / evaluation?}

\subfile{examples}

\section{Conclusion}

\subfile{conclusion}

\bibliographystyle{plain}
\bibliography{references}

\end{document}
