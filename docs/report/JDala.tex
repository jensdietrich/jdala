\documentclass[conference]{IEEEtran}
\IEEEoverridecommandlockouts
% The preceding line is only needed to identify funding in the first footnote. If that is unneeded, please comment it out.
%Template version as of 6/27/2024

\usepackage{cite}
\usepackage{amsmath,amssymb,amsfonts}
\usepackage{algorithmic}
\usepackage{graphicx}
\usepackage{textcomp}
\usepackage{xcolor}
\usepackage{subfiles}
\usepackage{todonotes}
\usepackage{listings}
\usepackage{xspace}
\usepackage{url}

\newcommand{\dala}{Dala\xspace}
\newcommand{\jdala}{JDala\xspace}
\newcommand{\Immutable}{\texttt{@Immutable}\xspace}
\newcommand{\Local}{\texttt{@Local}\xspace}
\newcommand{\Isolated}{\texttt{@Isolated}\xspace}
\newcommand{\portal}{\texttt{Portal}\xspace}

\def\BibTeX{{\rm B\kern-.05em{\sc i\kern-.025em b}\kern-.08em
    T\kern-.1667em\lower.7ex\hbox{E}\kern-.125emX}}


\lstdefinestyle{mystyle}{
	basicstyle=\ttfamily\scriptsize,
	breakatwhitespace=false,         
	breaklines=true,                 
	captionpos=b,                    
	keepspaces=true,                 
	numbers=left,                    
	numbersep=5pt,                  
	showspaces=false,                
	showstringspaces=false,
	showtabs=false,                  
	tabsize=2,
	frame=single
}

\lstset{style=mystyle}


\begin{document}
	
	

\title{JDala - A Simple Capability System for Java*\\
\thanks{TODO: @James.}
}

\author{
	\IEEEauthorblockN{Quinten Smit\IEEEauthorrefmark{1}, Jens Dietrich\IEEEauthorrefmark{1}, Michael Homer\IEEEauthorrefmark{1}, Andrew Fawcett\IEEEauthorrefmark{1}, James Noble\IEEEauthorrefmark{2}}
	\IEEEauthorblockA{\IEEEauthorrefmark{1}Victoria University of Wellington, School of Engineering and Computer Science, Wellington, New Zealand
		\\smitquin@myvuw.ac.nz,\{jens.dietrich,michael.homer,andrew.fawcet\}@vuw.ac.nz}
	\IEEEauthorblockA{\IEEEauthorrefmark{2}Creative Research and Programming, Wellington, New Zealand
		\\kjx@acm.org}
}


\maketitle

\begin{abstract}


\dala is a novel capability-based programming model that ensures data-race freedom while also supporting efficient inter-thread communication. While \dala has been designed to inform the design of future programming languages, the question arises whether existing languages can be retrofitted with \dala capabilities. We report such a design called \jdala. In \jdala,  \dala capabilities are added to Java using annotations and interpreted using bytecode instrumentation. With some examples we demonstrate that by adding three simple annotations to the language, we can avoid concurrency bugs like deadlocks and unexpected program behaviour resulting from shallow immutability of Java standard library APIs. 

\end{abstract}

\begin{IEEEkeywords}
object capabilities, java, instrumentation, concurrency, immutability 
\end{IEEEkeywords}

\documentclass[JDala.tex]{subfiles}
\graphicspath{{\subfix{../images/}}}
\begin{document}
%
Concurrent programming is ubiquitous today, driven by hardware advancements and software demands, but prone to subtle errors that often pass undetected and cause unexpected behaviour. Mainstream programming languages—especially those with shared-memory concurrency—leave developers vulnerable to concurrency bugs such as data races, race conditions, and deadlocks, which are notoriously difficult to detect, reproduce, and debug.
%
\dala~\cite{Dala_Paper,DafnyExperience-Noble2024} is a simple, novel model of object-oriented capabilities with dynamic enforcement. It allows parts of a program’s state to be protected from mutation (\texttt{imm}), concurrent access across threads (\texttt{local}), or aliased references (\texttt{iso}), thereby helping inhibit data races and related errors. These three capability flavours enforce constraints: immutability prohibits state change, isolation ensures thread exclusivity and non-aliasing, and locality guarantees thread affinity. Violations of these constraints are detected dynamically and halted with informative errors, enabling a fail-fast programming style that improves safety and ability to debug.
%
Work on \dala has been based on extensions to the Grace programming language~\cite{GraceAbsence-Black2012}, a simple, educational language designed with minimal concepts. However, this prior work has not addressed how the model might be applied to existing, complex, production-oriented languages with real-world concurrency requirements.
%
\jdala is an attempt to augment Java with \dala capabilities: to allow Java objects to be given one of the \dala flavours, and to detect and report violations of their constraints at run time. Unlike static approaches such as Rust’s affine type system or Pony’s actor-based isolation, \jdala uses lightweight annotations and bytecode instrumentation to retroactively enforce capability restrictions within standard Java programs. With only three core annotations, Java developers can sidestep common concurrency pitfalls—such as unsafe aliasing or illegal thread-sharing—while retaining the structure and familiarity of the Java ecosystem.
%
\jdala permits the programmer to mark some parts of the code with additional protections, while leaving other parts unchanged, enabling incremental adoption with minimal impact on existing code bases. As in many gradual typing systems, \jdala performs dynamic run-time checks to ensure that restrictions are not violated, and halts the program immediately if they are. This approach lets the program execute until a visible error occurs, then prevents the effects of that error from propagating further. This fail-fast behaviour assists developers in debugging and prevents concurrency bugs, like race conditions or misuse of immutability, from corrupting the run-time state of the program.
%
The contributions of this paper are:
\begin{itemize}
\item A design for \jdala, adding \dala capabilities to Java
\item A prototype implementation of \jdala using annotations and bytecode instrumentation
\item Examples of how \jdala can avoid concurrency bugs such as data races and deadlocks
\end{itemize}

\end{document}


\documentclass[../JDala.tex]{subfiles}
\graphicspath{{\subfix{../images/}}}
\begin{document}
	
	
\todo[inline]{@James @Michael}

\end{document}

\section{Motivating Examples}

\subsection{Immutable Sorted Lists}	


Consider the following code in Listing~\ref{listing:sort}~\footnote{The full source code of the examples used here can be found in the project repository: \url{https://github.com/jensdietrich/jdala/blob/main/jdala-core/src/test/java/nz/ac/wgtn/ecs/jdala/realWorldExamples/}. Examples are written as unit tests with oracles illustrating their behaviour with and without \jdala instrumentation.}. This code attempts to create an unmodifiable sorted list of \texttt{Person} instances. 
However, while \texttt{Collections::unmodifiableList} makes the list immutable, the objects within the list can still be mutated, including changes to the name property used as a sort key.  Once such mutations 
have taken place, it is no longer guaranteed that the members of the list are sorted by name, and an application that incorrectly relies on such an assumption may exhibit unexpected behaviour.

\begin{lstlisting}[language=Java, caption=Erroneous Attempt to Make a Sorted List Immutable, label=listing:sort]
	List<Person> people =  .. ;
	Collections.sort(people,
	Comparator.comparing(Person::getName));
	Collections.unmodifiableList(people);
	for (Person p:people)  System.out.println(p);
\end{lstlisting}



In \jdala ,  this can be prevented by annotating the respective list as \Immutable. This is shown in Listing~\ref{listing:sort-with-jdala}, line 4. In contrast to \texttt{Collections::unmodifiableList}, immutability is now deep, i.e. it also applies to all objects within the list. This is enforced dynamically (i.e. at runtime) by intercepting attempts to change the state of objects in the list.  Attempts to change the state of immutable objects are signalled with a \texttt{DalaCapabilityViolationException}. 

\begin{lstlisting}[language=Java, caption=Make a Sorted List Immutable with \jdala, label=listing:sort-with-jdala]
	List<Person> people =  .. ;
	Collections.sort(people,
	Comparator.comparing(Person::getName));
	@Immutable immutPeople=people;
	..
\end{lstlisting}


In the sorted list example, calls to \texttt{Person::setName} (which writes to the \texttt{Person::name} field) will now result in a runtime exception. 
This is \textit{fail-fast behaviour}~\cite{shore2004fail}, i.e. unexpected behaviour is avoided by producing an informative error signal at the point where the issue occurs. 

\subsection{Deadlock Prevention}	

Consider Listing~\ref{listing:deadlock}.  This is a simple method for transferring money between two accounts. To ensure that sufficient funds are available, the respective accounts are locked using the \texttt{synchronized} keyword.  However, if an application encounters a situation in which money has to be transferred within a short time window between two accounts in both directions, a deadlock can occur causing the application to stall~\footnote{In the full example code, this is illustrated by using a timeout oracle. The deadlock can be observed with a JMX client like VisualVM.}.


\begin{lstlisting}[language=Java, caption=Money transfer implementation prone to deadlock, label=listing:deadlock]
	void transfer(Account from, Account to, double amount) {
		synchronized (from) {
			from.withdraw(amount);
			Thread.sleep(1_000); // to simulate database write(s)
			synchronized (to) to.deposit(amount);
	}}
	.. 
	ThreadPoolExecutor tpool = .. ;
	@Local Account acc1,acc2  ..;
	Future f1 = tpool.submit(() -> transfer(acc1, acc2, 50)); 
	Future f2 = tpool.submit(() -> transfer(acc2, acc1, 80));
\end{lstlisting}

%Without \jdala, executing this program results in a deadlock.  
By  running this program with the \jdala agent and annotating the accounts with \Local , the deadlock can be prevented. The instrumented program immediately fails with a \jdala exception (wrapped in the \texttt{ExecutionException} referenced by the future instances) indicating that the \texttt{Account} instances are associated with the main thread, but used in different (thread pool) threads. 



\documentclass[../JDala.tex]{subfiles}
\graphicspath{{\subfix{../images/}}}
\begin{document}
	
\subsection{Motivational Example 1 - Immutability}	


Consider the following code in Listing~\ref{listing:sort}~\footnote{TODO: add reference to full code in anonymised repo}.

\begin{lstlisting}[language=Java, caption=Erroneous Attempt to Make a Sorted List Immutable, label=listing:sort]
List<Person> people =  .. ;
Collections.sort(people,
   Comparator.comparing(Person::getName));
Collections.unmodifiableList(people);
for (Person p:people)  System.out.println(p);
\end{lstlisting}

This is an attempt to create an unmodifiable sorted list of \texttt{Person} instances. 
However, while \texttt{Collections::unmodifiableList} make the list immutable, the objects within the list can still be mutated, including changes to the name property used as sort key.  Once such a mutation 
has taking place, it is no longer guaranteed that the members of the list are sorted by name, and an application that incorrectly relies on the assumption that sorting is in place will encounter unexpected behaviour.

In \jdala ,  this can be prevented by declaring the respective list as immutable. Since objects cannot be directly annotated in Java, a local variable pointing to the respective object is annotated instead.  This is shown in Listing~\ref{listing:sort-with-jdala}. 

\begin{lstlisting}[language=Java, caption=Make a Sorted List Immutable with \jdala, label=listing:sort-with-jdala]
List<Person> people =  .. ;
Collections.sort(people,
   Comparator.comparing(Person::getName));
@Immutable people2=people;
..
\end{lstlisting}

Once a variable is annotated, \jdala will mark the object it points to as immutable, and will prevent any further modifications of this object. 
In particular, calls of \texttt{Person::setName} (writing the field \texttt{Person::name}) will result in a runtime exception. 
This is \textit{fail-fast behaviour}~\cite{shore2004fail}, i.e. unexpected behaviour is avoided by creating an informative error signal at the point the issue is caused. 
 
\subsection{Motivational Example 2 - Deadlock Prevention}	

Consider Listing~\ref{listing:deadlock}.  This is a simple method to transfer ,money between two accounts. To ensure that there are sufficient funds available, the respective accounts are locked using the  synchronized keywords.  However, if an application encounters a situation where money has to be transferred within a shorty time distance between two accounts in both directions, a deadlock is created and the application stalls.

\begin{lstlisting}[language=Java, caption=Money transfer implementation prone to deadlock, label=listing:deadlock]
void transfer(Account from, Account to, double amount) {
	synchronized (from) {
		from.withdraw(amount);
		Thread.sleep(1_000); // to simulate DB workload
		synchronized (to) to.deposit(amount);
}}
\end{lstlisting}



	
\subsection{Java agents general intro}

\subsection{Annotation definitions}

Annotations

\subsection{Object Biased Definitions}

\subsection{signaling violations with runtime exceptions}

\subsection{global data structures to track annotated objects}

\subsection{memory issues}

\subsection{Dealing with Reflection}
(Field::set)

\subsection{Static Fields}

\subsection{modelling transfer}


\end{document}

% \section{examples / evaluation?}
\section{Usage and Evaluation}
\label{sec:useandevaluate}	


\subsection{Building and Using \jdala}

\jdala can be obtained from GitHub by cloning \url{https://github.com/jensdietrich/jdala/}.


\jdala requires two rounds of compiling, the first one to build the agent, and the second to build the code that is used. The first build creates the agent, but skips the tests:

\texttt{mvn clean package -DskipTests}

The reason is that most tests need the agent to function. Tests can be executed in a second build that uses the java agent created by the first build as value of the \texttt{-javaagent} JVM argument.

\subsection{Evaluation}


The tool is designed to be a technology readiness level 4~\cite{mankins1995technology} and has not been fully evaluated against real world data. Evaluation against relevant  benchmarks like \textit{Jacontebe}~\cite{lin2015jacontebe} is future work. This would require the manual annotation of benchmark code. There is however a comprehensive test suite included in the project.  



\section{Conclusion}

We have presented \jdala, a proof-of-concept implementation of the \dala capability model built atop the Java programming language.
The implementation demonstrates that novel programming language concepts can be retrofitted into existing mainstream languages, thereby promoting their broader adoption. The \jdala framework provides a platform for experimenting with capability-based security in the context of real-world applications."

 


\bibliography{references}
\bibliographystyle{plain}

\end{document}
