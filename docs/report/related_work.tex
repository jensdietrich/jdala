\section{Related Work}
	
\todo[inline]{@James @Michael}

\todo[inline,color=pink]{@James has added generic background stuff on
  concurrency cirectly below - needs to be trimmed hard.}


Concurrent programming is hard \cite{lea98,concurrentHard}
%
%Concurrent programming language design is even harder \cite{HoareHints}.
%
%JavaConcur
%Concurrent programming language design is even harder \cite{HoareHints}.
%
because concurrent programs do two or more things at the same
time. Even if one thread of execution is perfectly correct, another
thread can interfere with its behaviour corrupting data, triggering
deadlock, or crashing the whole system \cite{fonesca2010}.
%
%
The most important kind of interference between threads are
\textit{data races} --- when one thread writes to shared memory that
another thread is reading or writing. The problem with data races is
that values stored into memory, and the values returned by any reads,
may be essentially arbitrary, depending on the semantics of the
particular language, its implementation, and even the underlying CPU
and memory architecture!
%
%
%
%
%
This is exacerbated because concurrent programs are typically written in low-level languages
such as C, C$++$, BLISS-32, or Swift, which provide
few correctness guarantees, making it easy to write \textbf{wrong}
concurrent programs
\cite{CCppSCAM2014,LuSurvey2008}.
%\cite{stefikConcur2020,aba,raffi2020}
%
%% still like this, don't want to lose it!!!
%%
%% Suggestions that functional programming can solve all problems of
%% concurrent programming in the object-oriented mainstream have been
%% rebutted by pointing to a simple key problem: \textit{``assignment is
%%   faster than copying''} to paraphrase Doug Lea
%% \cite{AssignmentVSCopying}. In a world of mathematical abstractions,
%% we can (probably) ignore the physics and thermodynamics underlying our
%% programs: we can copy any amount of information any number of times on
%% any number of processors all for free, so we never need to mutate any
%% existing information, and we never have to face the problems of
%% synchronisation and inference between threads.
%% %
%% For the rest of us, our programs end up embodied in physical models
%% running on physical processors \cite{betabook}, where copying a
%% gigabyte of information unavoidably takes a million times as much work, uses
%% a million times more power, allocates a million times as much space,
%% and generates a million times more heat, than overwriting one
%% kilobyte. We cannot ignore the differences that leak out between our
%% logical, immutable maps, and their underlying implementation
%% territories of memory locations that store data and the threads that
%% animate them \cite{CopyExpensive}.
%
% SOB...
%
%Welcome to the desert of the real
%\cite{baurdillard,theMatrix,zizek}.
%
%

A range of programming techniques have been developed avoid these problems
\cite{lea98,concurrentHard}, aiming to provide \textit{safe
  concurrency} where data races are prevented either by the design  of the
programming language itself or its associated tools.  Unfortunately, while 
concurrent programming is hard, safe concurrent programming language
design turns out to be even harder \cite{RustBook,EncoreTS}. 
%
For example, full-scale proof systems
%\cite{vcc09}
such as concurrent separation logics\cite{JacobsEA05,chalice},
Rely/Guarantee\cite{jonesTOPLAS83,MPC-Staden15,concur2007} and
Deny/Guarantee\cite{DenyGuarantee} reasoning, and IRIS\cite{dd}
can guarantee correctness for programs in almost any languages,
but are too specialised for developers to use in practice
\cite{fonesca2017,shriramFormal2019}.

Rather more pragmatically,
contemporary programming languages such as
Rust\cite{RustBook}, Pony\cite{PonyTS},
Encore\cite{EncoreTS},
%Singularity\cite{Singularity},
%Deterministic Parallel Java\cite{DPJ},
%Safe C\verb+#+\cite{GordonPPBD12},
Obsidian\cite{aldrichObsidianStudy2020}, and
Verona \cite{Verona} 
have demonstrated the efficacy of \textit{static
  ownership}\cite{ClaPotNobOOPSLA98,NobPotVitECOOP98} %%% BoyLeeRinOOPSLA02,ClaPotNobOOPSLA98 
%and capability systems
to ensure
concurrent programs are safe: Rust in particular has been adopted by
Microsoft\cite{RustPopular,MSRust}. By keeping track of each object's
ownership, these languages can determine when an object may be used,
when it may be changed, and the effects those changes can have on the
rest of the program.
%
%
%% Concurrent programming languages face three problems:
%% %
%% \textbf{Correctness}: concurrent threads must only interact with each other
%% in well understood ways \cite{MPC-Staden15,concur2007,chalice}.
%% %
%% % interference between threads must be prevented.
%% %
%% \textbf{Simplicity}: programmers must be able to read, write, and understand
%% safe, efficient programs \cite{lea98,stefikConcur2020,godefroid2008}.
%% %
%% \textbf{Performance}: 
%% concurrent programs should run efficiently
%% (or else why not just write a non-current program\cite{TeslaRewrite,RustRewrite}).
%% %
%
%
While much simpler than full-scale program proof systems, these static
concurrent 
languages rely on complicated static (compile-time) rules and
restrictions, with many different capability annotations and ownership
parameterisations that programmers find hard to learn and use
correctly \cite{LearnRust,VizRust,HardRust}: they support writing
correct and efficient programs, but they are still hard to understand
\cite{SafeRust,FightRust} for a number of
reasons. First, their design must be conservative, banning not just
all concurrent programs that are \textit{actually} unsafe, but a large
number of data-race free programs as well.  To programmers, this
manifests as a large number of \textit{false positive} errors or
warnings about problems that will never arise in practice.  For
example, Rust's version of ownership types\cite{RustBook} bans even
such common idioms such as circular or doubly-linked lists.  Second,
programmers typically have to annotate their programs to give the
ownership and capability checkers the information they need --- so
rather than declaring an input stream ``\verb+in+'', programmers need
to write complex expressions such as ``\verb+in : &mut InStream<'a>+''
(where ``\verb+&mut+'' indicates that ``\verb+in+'' is a mutable
reference, ``\verb+InStream+'' indicates that ``\verb+in+'' refers to
an input stream, and  ``\verb+<'a>+'' is a lifetime (aka ownership)
parameter indicating the originating scope of the input stream.
Finally, these ownership annotations are required throughout
the program,  even if only a small part is actually concurrent, or is
otherwise likely to cause critical errors --- in Rust, an inflight
entertainment system would have to be engineered to the same level of
quality as a critical flight control system, even though the risks and
requirements for each system are very different.


%how to bring in actors etc?
%not sure about this bit
%
%
%
%% Application programmers are increasingly moving towards
%% \textit{dynamic} languages such as Python, R, and Julia
%% \cite{dynamic}.  These languages make it easier to write sequential
%% code, but generally provide no more support for wiring concurrent
%% code than sequential systems programming languages (and are are much
%% less efficient).  The Julia documentation, for example, says that
%% \textit{`` You are entirely responsible for ensuring that your program
%%   is data-race free, and nothing promised here can be assumed if you
%%   do not observe that requirement. The observed results may be highly
%%   unintuitive ''}
%% \cite{https://docs.julialang.org/en/v1/manual/multi-threading/}
%% \textsf{Q - do I want to say ``data race freedom?''.  or what?.}
%

On the other hand,
%meanwhile
\textit{dynamically} concurrent languages like
JavaScript\cite{JSVAts},
%Erlang\cite{Erlang},
E\cite{MillerPhD}, and
AmbientTalk\cite{AmbientTalk} support simple,
concurrency-safe programming by design, ruling out whole classes of
bugs.
Programmers don't need to annotate their code
(an input stream could be just declared as ``\verb+in+'')
making programs easier to read and write.
Unfortunately, these kinds of dynamic approaches achieve safety and simplicity
by sacrificing performance, in particular by banning efficient
communication between concurrent threads.
It is not possible to transfer ownership of an object:
instead, the objects must either be copied
\cite{destructive-read}
%
%(which while simplifying
%garbage collection may be expensive \cite{CopyExpensive} and loses
%object identity),
(taking significant time and memory, and losing object identity)
or proxied back to their originating thread
\cite{CASTEGREN2018130,JSproxies,AmbientTalk}
%(which makes it hard to reason about performance unless it is clear
%that an operation is asynchronous).
(greatly delaying every access from all other threads).
Some recent dynamically concurrent research langauges do e.g.\ permit
information to be transferred or shared directly between threads,
but either carry out runtime checks to ensure threads do not interfere
\cite{Daloze2016,Daloze2018}
or omit those checks, permitting data races, and giving up on safety
\cite{GoConcurBugs2019,raffi2020}.


