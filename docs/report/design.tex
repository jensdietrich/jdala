\section{The Design of \jdala}
\label{sec:design}	


\subsection{Capabilities as Annotations}
\label{subsection:capabilitesAsAnnotations}

In \jdala, the \dala capabilities are represented using three dedicated annotation types \Immutable, \Isolated and \Local.  Global data structures (maps that are static members of the \texttt{JDala} class) are used to track objects with the respective capabilities.   

Since in Java  objects cannot be annotated directly, the association between objects and object capabilities is achieved by annotating local variables pointing to objects.  This is illustrated in Listing~\ref{listing:annotation}.  In line 1, the newly created \texttt{List} instance is marked as immutable.  A second list created in line 2 is not marked as immutable at the allocation site, but later. This is achieved by an assignment to the annotated variable \texttt{list3}. 


\begin{lstlisting}[language=Java, caption=Associating objects with capabilities, label=listing:annotation]
@Immutable List list = new ArrayList();
List list2 = new ArrayList();
@Immutable List list3 = list2;
\end{lstlisting}


\subsection{Enforcing the Semantics of Capabilities}

The semantics of the respective capabilities are implemented using instrumentation that injects code to enforce them. 
\jdala uses ASM~\cite{bruneton2002asm} for this purpose, and the project builds an agent that can then be attached to any Java application either statically (using the \texttt{-javaagent} argument) or dynamically.
The instrumentation controls the heap of an instrumented application by maintaining safe objects~\footnote{\textit{Safe} objects are objects annotated using either \Immutable, \Isolated or \Local, all other objects are referred to as \textit{unsafe} objects.} in global data structures, and intercepting and checking bytecode instructions reading and writing fields (i.e. traversing and manipulating the heap).  Instructions for reference type fields, special instructions for fields with one of the various Java primitive types, and instructions to access array slots are all instrumented. 

The injected code broadly falls into two categories: \textit{registration} and \textit{enforcement}. When annotated references to objects are encountered, the corresponding objects are registered as immutable, local or isolated in static maps maintained by the \texttt{JDala} class. In case of immutability, referenced objects are registered as immutable as well. For this purpose, a simple reflection-based heap traversal is performed on the object to be registered as immutable.  If already registered objects are re-registered with a weaker capability~\footnote{The capabilities considered here form a hierarchy, see \cite{Dala_Paper} for details}, a \texttt{DalaRestrictionException} is raised.  This functionality is implemented in various \texttt{JDala:register*} methods.

When fields of registered objects are accessed, the injected code invokes check methods \texttt{JDala:validate*} to enforce the capability contract. Violations are signalled by raising a \texttt{DalaCapabilityViolationException} exception.


\subsection{Object vs Class-Based Capabilities}


\jdala provides two mechanisms to define the capabilities. The primary method uses the object-level annotations discussed in Section~\ref{subsection:capabilitesAsAnnotations}. This method assigns a capability to an object when a variable pointing to it is annotated. Any future objects that are stored in the local variable after the object that has been annotated must once again have an annotation present to be assigned a capability or they will be considered unsafe.
Once an object has been assigned a capability, it retains that capability—or a stricter one—for its entire lifetime.

A secondary mechanism applies only to \Immutable capabilities and involves defining immutable classes globally. Classes listed in \texttt{resources/immutable-classes.txt} are automatically treated as immutable. This approach is suitable for Java classes that are intrinsically immutable, such as \texttt{String}, \texttt{Boolean}, \texttt{Integer}, and \texttt{Byte}. Note that all primitive types are treated as \texttt{@Immutable} by default.

Recognizing these classes as \Immutable is essential, as it enables them to be safely included within \Local or \Isolated objects without requiring explicit annotation. Without this class-wide designation, such intrinsically immutable types—despite their known immutability —would otherwise be treated as unsafe, potentially restricting their use in contexts where immutability is a requirement.

% thanks @Quinten, this work now ! 


\subsection{Object Initialisation Protocol}

Constructors provide two unique challenges. The first is caused by a Java bytecode optimisation that allows object fields to be set before an object's constructor is called. This isn't allowed in Java source code \footnote{Except JEP447 (\url{https://openjdk.org/jeps/447}) which allows statements before \texttt{super(...)}, JEP447 is current at preview stage.} However, compilers  can still generate such bytecode. At this early stage in the constructor, the object has not yet completed initialisation and does not fully extend  \texttt{Object}. In the Java bytecode, this incomplete state is represented using the special \texttt{UninitializedThis} value instead of the standard \texttt{this} reference. As a result, it cannot yet be treated as a fully valid object, which complicates instrumentation and capability tracking during construction.
To deal with this special case, \jdala 's instrumentation checks whether field access occurs in a constructor, and if so will temporary store values to be checked in local variables. As soon as the super-constructor is called, \jdala will perform all of the validation checks at once for those values. This means that in some cases the line numbers in stack traces in \jdala exceptions created when those checks fail might be off by a few lines. 

% J > Q : I am not sure whether this adds a lot of value, but feel free to bring it back if you feel it does
%Importantly the amount of actions that can be performed in a constructor and before a super constructor is very limited, so errors will still occur before the object can properly used.

The second challenge is related to immutable classes. Unlike annotated objects, which are registered as immutable only after their construction, globally defined immutable classes are considered immutable by default and are therefore subject to capability enforcement from the outset. This creates a complication: during construction, these classes must be allowed to modify their internal state, but doing so must not compromise the overall soundness of the system, particularly with respect to other objects accessed within the constructor.
To address this, \jdala introduces a targeted exception to its enforcement rules: it permits an object to modify its own fields within its own constructor, regardless of whether it is marked as \texttt{@Immutable}. This exemption allows immutable objects to be initialized correctly while preserving capability safety for interactions with other objects during construction.
%\todo[inline]{J $>$ Q: can you please add a short paragraph here ? We do have space for a short code listing. }
%\todo[inline, color=cyan]{Q $>$ J: Sorry didn't quite keep it short, it might be able to be trimed a little bit by doing less detail}
% J>Q: great, I did some cosmetic changes



\subsection{Static Fields}
Static fields are shared across all instances of a class and are therefore considered class-level, rather than object-level. Modifications to static fields are not captured by \jdala's object instrumentation. 
%As such, developers must take extra care when working with static data in the context of \dala capabilities.

\subsection{Memory Leak Prevention}

A key implication of using global data structures for capability tracking is the potential for memory leaks. Because references are stored in maps accessed through static variables, those objects may become eligible for garbage collection. To mitigate this, weak references are employed within these collections, allowing unused objects to be reclaimed by the JVM's garbage collector when no strong references remain.


\subsection{Reflection Support}

Java reflection can be used to bypass conventional field access patterns, posing a challenge to capability enforcement. To address this, \jdala instruments the \texttt{Field::get} and \texttt{Field::set} methods. This ensures that field modifications performed via reflection still trigger the appropriate capability checks.

There are other dynamic programming patterns that could be used to bypass the capability contracts. In particular, \jdala does not currently restrict the use of \texttt{Unsafe}~\cite{mastrangelo2015use}. However, direct use of \texttt{Unsafe} by applications is discouraged and restricted in newer versions of Java.


\subsection{Modelling the Transfer of \Isolated objects}

\dala allows for isolated objects to be transferred between threads.
To move isolated objects between threads \dala outlines the use of a consume method that allows only one reference to an object. 
Any isolated object that moves from one thread to another will lose its affiliation with the first thread. This principle can be recreated in Java by setting a reference to \texttt{null} after an object has moved. This could lead to a non-descriptive \texttt{NullPointerException} later in the code.

For \Isolated objects \jdala allows multiple references to exist in one thread, with checks being performed that the object remains in its associated thread. \Isolated objects are initially associated with the thread they were created in. For transferring an \Isolated object to a new thread a dedicated \textit{portal object} must be used. Portal object protocols have to be defined by specifying methods for objects to enter a portal (therefore being disconnected from the current thread) and methods to exit a portal (therefore becoming associated with the current thread).  While it is possible to define portal objects by annotating objects, with properties to define the respective enter and exit methods, we deemed this too complex. Most transfers follow patterns where portal objects are instances of dedicated \textit{portal classes} like blocking queues.  
These classes are defined in the resource \texttt{portal-classes.json}, a classical example is\texttt{ java.util.concurrent.BlockingQueue} with \texttt{put/\-offer/\-add} entry  and \texttt{poll/\-take/\-remove} exit methods. Defining a particular type as a portal also applies to its subtypes. 

If an isolated object enters a portal it goes into a \textit{transfer state}. In this state the object cannot be accessed \footnote{Object fields can neither be read nor mutated.} by any thread. The object can then leave the portal via one of the predefined exit methods and at this point, it leaves the transfer state and becomes owned by the thread invoking the exit method.


\subsection{Instrumentation Scope and Shading}

Common challenges when building agents are self-instrumentation and instrumentation of internal utility classes, such as the data structures used to track capability-annotated objects. Instrumenting these classes directly could result in unintended recursive instrumentation cycles, potentially compromising runtime stability. To address this, \jdala uses \textit{shading} to incorporate private, namespaced copies of \texttt{org.json}, \texttt{org.plumelib.util}, and a concurrency map wrapper from \texttt{java.util.Collections}. These shaded versions are explicitly excluded from instrumentation. To maintain soundness, these shaded classes is restricted exclusively to the \texttt{JDala} class and should not be accessed by other parts of the application. 

In addition, several \texttt{--add-opens} flags have been introduced to the project's Maven configuration. These flags explicitly grant reflective access to internal Java platform modules that are otherwise inaccessible under the Java Module System. This access is essential for \jdala to perform bytecode instrumentation, intercept field and method accesses, and monitor runtime behaviour across a wide range of classes. Without these flags, the agent would encounter \texttt{IllegalAccessException} at runtime, or fail to apply the necessary instrumentation to enforce capability semantics reliably.

%\subsection{Java Bytecode Instrumentation}
%\jdala employs the ASM framework to instrument Java bytecode at runtime. This allows interception of field and array access operations at the bytecode level. Specifically, instructions like \texttt{getField}, \texttt{putField}, and array access opcodes are intercepted and augmented with capability checks.
